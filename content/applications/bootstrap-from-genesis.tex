\subsection{Bootstrapping from the Genesis Block}
\label{subsec:bootstrapping-from-genesis}

In this section, we focus on the \emph{observability} property of our SMR protocol.
%
Recall that in~\cref{subsec:phase-oblivious-agreement}, we stated that a full agreement on all parallel chains in the previous phase is impossible, and parties that join at a specific phase cannot learn the previous execution by ``tracing back'' using cross-chain reference.
%
Thus, it becomes challenging or even impossible for a passive observer to join the protocol in the middle of the execution.
%
To solve this, in this section we slightly modify our Chain-King Consensus protocol and design a bootstrapping algorithm for fresh parties to synchronize state with all honest parties.
%
Note that the design of a bootstrapping procedure to let fresh parties join is also an essential building block for protocols that support dynamic participation.

When a fresh party \newParty joins, \newParty has no knowledge about the protocol execution except for the CRS and global time (recall that we assume synchronous processors).
%
To become synchronized and learn the ledger state, \newParty needs to bootstrap by passively listening to the protocol.
%
We highlight that, in order for \newParty to synchronize with other honest parties (i.e., achieving phase oblivious agreement), \newParty needs to run a bootstrapping procedure which lasts for a constant number of rounds (precisely \phaseLength rounds).

In order to let fresh parties join the protocol, we modify our Chain-King Consensus protocol as follows.
%
In the $i$-th phase $(i > 1)$, concatenated with the consensus-related input message, parties also include the fresh randomness extracted from their local chains in the $(i - 1)$-th phase.
%
More specifically, they extract the hash of the last block in the output generation stage on each chain in the $(i - 1)$-th phase of \parallelChainsLocal, assemble them as a $\kappa$-bit string and append it to the input-block content.
%
For chains where a typical execution holds, honest parties adopt the same block hash.
%
Next, in $i$-th phase, a Crusader Agreement is  run on the block hash of each chain in the $(i - 1)$-th phase (recall from~\cref{remark:crusader-agreement} that a single phase suffices to serve as a Crusader Agreement protocol).
%
I.e., for the $j$-th chain with a typical execution, parties agree on a unique block hash that is the same as their local $\parallelChainsIdx{i - 1}_j$, and for other chains, all parties either output the same hash or $\bot$.

Thus, when a fresh party \newParty joins the protocol, she first passively listens to the protocol for \phaseLength rounds so that she observes the end of a phase, say phase $i$.
%
Our chain selection rule (\cref{algorithm:update-local-chain}) guarantees that \newParty has parallel chains in phase $i$ that obliviously agree with other honest parties on more than $3m/4$ chains (recall \cref{thm:phase-oblivious-agreement}).
%
Now, \newParty can ``trace back'' all the chains where typical execution holds by using the fresh randomness included in the current phase; and iterate them phase-by-phase.
%
Specifically, when \newParty is at the end of phase $i$, she runs \cref{algorithm:joining-procedure} to extract the hashes of dense chains in the previous phase and use them to form her local chain $\parallelChainsLocal^{(i - 1)}$.
%
For instance, consider the $j$-th chain in the $(i - 1)$-th phase.
%
If on more than $3m/4$ chains in phase $i$, a majority of the input blocks report fresh randomness that matches a chain $\chain \in \denseChains[i - 1][j]$, then \newParty will select \chain and add it as the $j$-th chain in $\parallelChainsLocal^{(i - 1)}$.
%
If no such chain exists, \newParty will randomly pick a chain or just leave it empty.

\begin{cccAlgorithm}
    {\textsf{JoiningProcedure}}
    {joining-procedure}
    {The joining procedure for a fresh party.}

    \begin{algorithmic}[1]
        \For{$i = 1$ \textbf{to} \phaseLength}
        
        \State{Fetch information and dentoe the incoming chains and input-blocks by $\langle \parallelChains, denseChains \rangle_1,\allowbreak \ldots, \langle \parallelChains, denseChains \rangle_n$ and $\IB_1, \ldots, \IB_{n'}$}
        
        \State{Add $\langle \parallelChains, denseChains \rangle_1, \ldots, \langle \parallelChains, denseChains \rangle_n$ to \chainBuffer}
        
        \State{Add  $\IB_1, \ldots, \IB_{n'}$ to \inputBlockBuffer}
        
        \State{$\parallelChainsLocal \gets \mathsf{UpdateLocalChain}(\parallelChainsLocal, \langle \parallelChains, denseChains \rangle_1, \ldots, \allowbreak \langle \parallelChains, denseChains \rangle_n)$}

        \If{$\round \mod \phaseLength = 1$} \Comment{Retrieve all previous phase}
        
        \For{$i$ \textbf{from} $\phase - 1$ to $1$}
        
        \For{$j = 1$ \textbf{to} $m$}
        
        \State{$\chain \gets \parallelChainsIdx{i + 1}_j$}
        
        \State{Initialize $\vec{h}$ to an empty array}
        
        \For{$q = 1$ \textbf{to} $m$}
        \State{Initialize $\vec{M}$ to an empty array}
        \For{$\block \in \chain $ \textbf{and} $tp \cdot \phaseLength - (\phaseOutputLength + \phaseRefLength) < \timestamp{\block} \le tp \cdot \phaseLength - \phaseRefLength$}
        \State{Parse chain pointer to last phase as $h'$}
        \State{Append $\stringSegment{h'}{q}{m}$ to $\vec{M}$}
        \EndFor
        \State{Sort $\vec{M}$ then append $\med(\vec{M})$ to $\vec{h}$}
        \EndFor
        
        \State{Let $h$ denote the most frequent element in $\vec{h}$ and $c$ its frequency}
        
        \If{$c >3m/4$ \textbf{and} $\exists \chain \in \denseChains[i][j]$ such that $\stringSegment{\chainHead{\chainPrefix{\chain}{\phaseRefLength}}}{j}{m} = h$}
        \State{Add \chain as the $j$-th chain in $\parallelChainsLocal^{(i)}$}
        \Else
        \State{Choose a chain $\chain \in \denseChains[i][j]$ randomly as the $j$-th chain in $\parallelChainsLocal^{(i)}$ (or if no dense chain, select one from incoming $\parallelChainsIdx{i}_j$ randomly)}
        \EndIf
        
        \EndFor
        
        \EndFor
        
        \EndIf
        
        \EndFor
        
        \State{Call $\mathsf{StateUpdate}$ on each phase to build the ledger}
        
        \State{Set synchronized with all honest parties}
    \end{algorithmic}
\end{cccAlgorithm}


Note that the security of both Chain-King Consensus and Crusader Agreement only rely on the consistent view of chains where typical execution holds; hence, at the end of the joining procedure, \newParty achieves phase oblivious agreement with all honest parties.
%
As a result, \newParty can reconstruct the entire execution and update her internal state to build the whole ledger.
