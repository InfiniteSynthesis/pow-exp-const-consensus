\section{Algorithms Omitted in the Main Body}
\label{sec:algorithms-omitted}

\paragraph{Chain validation check.}
%
The following functions help us simplify the validation process.
%
First, we adopt $\mathsf{ValidContent}$ to validate the block content.
%
When the input is \IB and $\mathsf{ValidContent}$ extracts its associated block content and returns true only when the block content is a valid type of input in our consensus protocol; when the input is \block, $\mathsf{ValidContent}$ returns true only when the associated block content contains only block headers of input-blocks and other types of valid transactions (we omit the details as it is irrelevant to our consensus protocol).

Next, we use $\mathsf{ValidBlock}^T$ to verify if a (chain-)block is a successful PoW on the $i$-th chain (that is, the nonce $ctr$ is valid and the block hash --- $i$-th segment of the RO output is less than target $T$).
%
\[ \mathsf{ValidBlock}^T(\langle ctr, r, h, st, h', val \rangle,i) = \stringSegment{H(\langle ctr, r, h, st, h', val \rangle)}{i}{m} < T \wedge ctr < 2^{32} \]
%
Similarly, we use use $\mathsf{ValidInputBlock}^T$ to verify if an input-block is a successful PoW on the $i$-th chain by checking the reverse of the string segment.

Let $\mathsf{isDenseChain}$ denote the a predicate that returns \true iff. a chain \chain is a dense chain.
%
Slightly abusing the notations, when $\mathsf{ExtractInputFreshness}$ is called with a single chain \chain we let this denote the fresh randomness for input-blocks extracted from this chain (note that this can be computed without parallel chains).

\begin{cccAlgorithm}
    {$\mathsf{IsValidChain}(\parallelChains, denseChains, r)$}
    {isvalidchain}
    {The chain validation procedure.}

    \newcommand{\goodHash}{\mathEnv{\mathsf{goodHash}}}
    \newcommand{\goodNonce}{\mathEnv{\mathsf{goodNonce}}}
    \newcommand{\goodTime}{\mathEnv{\mathsf{goodTime}}}
    \newcommand{\goodContent}{\mathEnv{\mathsf{goodContent}}}
    \newcommand{\goodRef}{\mathEnv{\mathsf{goodRef}}}

    \begin{algorithmWithNumbering}{isValidChainCounter}
        \LineComment{This algorithm has five internal Boolean variables \goodHash, \goodNonce, \goodTime, \goodContent and \goodRef, all initialized as \true.}

        \For{$i$ \textbf{from} $1$ \textbf{to} \phase}
        \For{$j$ \textbf{from} $1$ \textbf{to} $m$}

        \State{Call $\mathsf{chainValidation}(\parallelChains, denseChains, r, i, j, \parallelChainsIdx{i}_j)$}

        \For{$\chain \in denseChains[i][j]$}
        \State{Call $\mathsf{chainValidation}(\parallelChains, denseChains, r, i, j, \chain)$}
        \oneLineIf{\textbf{not} $\mathsf{isDenseChain}(\chain)$}{$\goodRef \gets \false$}
        \EndFor

        \EndFor
        \EndFor

        \State{\Return $\goodHash \wedge \goodNonce \wedge \goodTime \wedge \goodContent \wedge \goodRef$}
    \end{algorithmWithNumbering}

    \medskip

    \begin{algorithmWithNumbering}{isValidChainCounter}
        \Statex{\underline{\textbf{Procedure} $\mathsf{chainValidation}(\parallelChains, denseChains, r, phase, chain, \chain)$}}
        
        \State{$r^* \gets \min \{r, phase \cdot \phaseLength \}$}
        
        \For{$k$ \textbf{from} $\chainLength{\chain}$ \textbf{to} $1$}
        
        \State{Parse $\chain_k$ as $\block = \langle ctr, r, h, st, h', val \rangle$}
        
        \If{$k > 1$}
        \State{Parse $\chain_{k - 1}$ as $\block^*$}
        \oneLineIf{$\stringSegment{h}{chain}{m} \neq \stringSegment{H(\block^*)}{chain}{m}$}{$\goodHash \gets \false$}
        \ElsIf{$\phase = 1$}
        \oneLineIf{$h \neq 0^{\kappa}$}{$\goodHash \gets \false$}
        \EndIf

        \oneLineIf{\textbf{not} $\mathsf{ValidBlock}^T(\block, j)$}{$\goodNonce \gets \false$}
        
        \oneLineIf{\textbf{not }$(phase - 1)\phaseLength < \timestamp{\block} < r^*$}{$\goodTime \gets \false$}
        
        \oneLineIf{\textbf{not} $\mathsf{ValidContent}(\block)$}{$\goodContent \gets \false$}
        
        \State{Run $\mathsf{IsValidInputBlocks}(\block, \chain, phase, chain)$}
        \State{$r^* \gets \timestamp{\block}$}

        \If{$phase > 1$}
        \Comment{Check cross-chain reference}
        
        \State{$denseCnt \gets 0$}
        \For{$q$ \textbf{from} $1$ \textbf{to} $m$}
        \State{Parse $\chain = \parallelChainsIdx{phase - 1}_{chain}$}
        \If{$\stringSegment{\chainHead{\chainPrefix{\chain}{\phaseRefLength}}}{q}{m} = \stringSegment{h'}{q}{m}$ \textbf{and} $\mathsf{isDenseChain}(\chain)$}
        \State{$denseCnt \gets denseCnt + 1$}
        \EndIf
        \If{$\exists \chain' \in denseChains[phase][chain]$ \textbf{and}$ \stringSegment{\chainHead{\chainPrefix{\chain'}{\phaseRefLength}}}{q}{m} = \stringSegment{h'}{q}{m}$}
        \State{$denseCnt \gets denseCnt + 1$}
        \EndIf
        \EndFor
        
        \oneLineIf{$denseCnt < \beta \cdot m$}{$\goodRef \gets \false$}
        
        \EndIf
        
        \EndFor
    \end{algorithmWithNumbering}

    \medskip

    \begin{algorithmWithNumbering}{isValidChainCounter}
        \Statex{\underline{\textbf{Procedure} $\mathsf{IsValidInputBlocks}(\block, \chain, phase, chain)$}}
        
        \For{each $\IB = \langle ctr, r, aux, h', val \rangle \in \block$}
        
        \If{$\exists \IB' \in \block' \in \chain$ \textbf{and} $\stringSegmentRev{H(\IB')}{chain}{m} = \stringSegmentRev{H(\IB)}{chain}{m}$}
        \State{$\goodHash \gets \false$}
        \EndIf
        \oneLineIf{\textbf{not} $\mathsf{ValidInputBlock}^T(\IB, chain)$}{$\goodNonce \gets \false$}
        \oneLineIf{\textbf{not} $(phase - 1) \phaseLength <r < \timestamp{\block}$}{$\goodTime \gets \false$}
        \oneLineIf{\textbf{not} $\mathsf{ValidContent}(\IB)$}{$\goodContent \gets \false$}
        \State{$h^* = \mathsf{ExtractInputFreshness}(\chain)$}
        \oneLineIf{$\stringSegment{h'}{chain}{m} \neq \stringSegment{h^*}{chain}{m}$}{$\goodRef \gets \false$}
        
        \EndFor
    \end{algorithmWithNumbering}
\end{cccAlgorithm}
