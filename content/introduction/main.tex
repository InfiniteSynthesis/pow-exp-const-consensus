\section{Introduction}
\label{sec:introduction}

Byzantine agreement (BA, aka consensus) is a classical problem introduced in~\cite{JACM:PeaShoLam80} that asks $n$ parties to agree on a message so that three properties are satisfied: (i) termination, (ii) agreement and (iii) validity, in a setting where any $t$ of the parties may behave maliciously.
%
Validity enforces the non-triviality of solutions, as it requires that if the non-faulty/``honest'' parties start the execution with the same value, then that should be the output value.

BA has been classically considered in a ``permissioned setting'': the parties running the protocol are setup so they are able to reliably and directly communicate with each other, or have access to a public-key directory that reliably lists all their public keys.
%
This is captured by a suitable network or \emph{trusted setup} assumption.
%
The ``permissionless setting,'' on the other hand, was introduced with the development of the Bitcoin blockchain \cite{Nak08}, and refers to an environment where parties may enter the protocol execution at will, the communication infrastructure is assumed to deliver messages without reliably identifying their origin, and the trusted setup is reduced to the existence of an unpredictable public string---the ``genesis block'' (which sometimes for simplicity we will just refer to as a CRS [common reference string], or ``public-state setup''~\cite{RSA:GarKia20}).

BA in the permissionless setting above using proofs of work (PoW)\footnote{As implemented in the Bitcoin  blockchain, via hash functions modeled as a \emph{random oracle} (RO)~\cite{CCS:BelRog93}.} was first (formally) studied in~\cite{EC:GarKiaLeo15}.
%
In terms of running time, the protocols presented in \cite{EC:GarKiaLeo15} run in $\bigO(\polylog \kappa)$ rounds, where $\kappa$ is the security parameter and address the binary input case, where the parties wish to agree on a single bit.
%
Subsequent work improved on various aspects at the expense of stronger assumptions.
%
For example, Andrychowicz and Dziembowski~\cite{C:AndDzi15} offered a multi-valued BA protocol also based on PoWs (RO) but with no trusted setup, assuming in addition the existence of existentially unforgeable signatures, and with a running time proportional to the number of parties.
%
The latter was in turn improved by Garay \textit{et al.} \cite{PKC:GKLP18} to $\bigO(\polylog \kappa)$ rounds, and just assuming PoWs and no trusted setup.
%
Recently, an expected-constant-round BA protocol was introduced by Das \textit{et al.}~\cite{EPRINT:DEFLM22}, by requiring in addition to the Andrychowicz and Dziembowski~\cite{C:AndDzi15} assumptions the existence of \emph{verifiable delay functions} (VDFs)~\cite{C:BBBF18}.
%
Refer to~\cref{table:PoW-BA} for a comparison of existing PoW-based (or ``PoW-inspired'') BA protocols.

\begin{table*}[ht]
    \begin{tabularx}{\linewidth}{@{\hskip .25in} c @{\hskip .5in}  c @{\hskip .5in} c @{\hskip .25in}}
        \toprule
        \textbf{Protocol}
         & \textbf{Setup \& assumptions }
         & \textbf{Round complexity}
        \\ \midrule
        \cite{C:AndDzi15}
         & RO + SIG
         & $\bigO(n)$
        \\ \midrule
        \cite{EC:GarKiaLeo15}
         & CRS + RO
         & $\bigO(\polylog \kappa)$
        \\ \midrule
        \cite{PKC:GKLP18}
         & RO
         & $\bigO(\polylog \kappa)$
        \\ \midrule
        \cite{EPRINT:EckFauLos17}
         & RO + SIG + TLP
         & Expected $\bigO(1)$
        \\ \midrule
        \cite{EPRINT:DEFLM22}
         & RO + SIG + VDF
         & Expected $\bigO(1)$
        \\ \midrule
        This paper
         & CRS + RO
         & Expected $\bigO(1)$
        \\ \bottomrule
        \caption{Round complexity of PoW-based (or PoW-inspired) permissionless Byzantine agreement protocols, with their corresponding setup and cryptographic assumptions.}
        \label{table:PoW-BA}
    \end{tabularx}
\end{table*}


Given the above state of the art, in this work we focus on the question of solving permissionless BA in the original PoW-based blockchain model of Bitcoin with expected-constant round complexity.

\subsection{Overview of Our Results}
\label{subec:overview-of-our-results}

We present a new permissionless PoW-based multi-valued BA protocol that has expected-constant round complexity and demonstrate how it can be used to solve permissionless \emph{state machine replication} (SMR, or, equivalently, a \emph{distributed ledger})~\cite{CSUR:Schneider90} with fast settlement.
%
In more detail, our results are as follows.

\smallskip\noindent\emph{A new PoW-based permissionless consensus protocol.}
%
We put forth \emph{Chain-King Consensus}---the first PoW-based permissionless consensus protocol that achieves agreement and validity in \emph{expected-constant} time.
%
Our construction is based on mining on parallel chains, and ``emulating'' a classical ``phase-king'' consensus protocol~\cite{FOCS:BerGarPer89} with a randomized chain (the ``chain-king'') selection rule on top of the parallel chains construction.
%
Our protocol is based on the following ideas.

First, we revisit the parallel chain technique (cf.~\cite{EPRINT:FGKR18,CCS:BKTFV19,TCC:FGKR20}) as a method for combining multiple blockchains advancing in parallel.
%
Our key observation is that running $m = \polylog(\kappa)$ parallel chains is sufficient to maintain independence via an \mforone\footnote{Pronounced ``$m$-for-$1$.''} PoW technique~\cite{EC:GarKiaLeo15} (while prior work set $m = \Theta(\kappa)$ and hence at best was only able to argue ``sub-independence''; see \cite{TCC:FGKR20}).
%
In fact, our protocol runs $m$ independent instances of \twoforone PoWs, with the latter component being responsible for transaction processing.\footnote{As in~\cite{EC:GarKiaLeo15}, the ``transactions'' being processed in a BA protocol are the input values being proposed by the parties.}
%
The key property we utilize is that in a constant number of rounds, a fraction of the $m$ parallel chains will be sufficiently advanced to offer a form of ``common prefix'' property (cf.~\cite{EC:GarKiaLeo15}) with a constant probability of success.

Second, and contrary to prior work on parallel chains, we ``slice'' the chain progression into stages where parallel chains can cross-reference each other.
%
In the first stage, parties converge on their views and ensure fresh randomness is introduced; in the second stage they process transactions; and in the third, they prepare for the cross referencing by the upcoming stage, after which the stages rotate indefinitely.
%
A key property of our cross-referencing rule is the concept of a \emph{dense chain}---a strengthening of the concepts of ``chain growth'' and ``chain quality''~\cite{EC:GarKiaLeo15}.
%
Given the short length of each stage (a constant number of rounds), chain density ensures that the adversary faces difficulties to create multiple compromised chains.
%
The key conclusion of this chain structure is \emph{phase-oblivious agreement}, which refers to the fact that, on a large fraction of chains, the majority of input values are contributed by honest parties.

The core agreement component of our protocol follows the ``phase king'' approach (cf.~\cite{ALP:BerGar89,FOCS:BerGarPer89}).
%
The key idea of porting this protocol design technique to the permissionless setting is to map the chains in the parallel chains cluster to the roles of the different parties in the classical protocol.
%
As a result, the king itself is one of the chains.
%
Moreover, due to the ``dilution'' of adversarial power that occurs in the parallel chains setting, we can set the king \emph{deterministically} to be a specific chain.
%
This technique, which may be of independent interest, results in our ``Chain-King Consensus'' algorithm.

Chain-King Consensus is one-shot, in the sense that it will provide just a single instance of agreement in the permissionless setting in expected-constant time.
%
The natural question given such protocol is whether it is possible to apply sequential self-composition with running time remaining expected linear in the number of instances.
%
This is a delicate task due to non-simultaneous termination (cf.~\cite{C:CCGZ16}).
%
We provide a round-preserving sequential composition solution that first adapts Bracha termination \cite{PODC:Bracha84} to the permissionless setting and reduces the ``termination slack'' among honest parties to 1 phase.
%
Then, we adapt the super-phase expansion technique of~\cite{C:CCGZ16} to widen the interval between state updates from 1 phase to 4 phases.
%
We identify a set of good properties for a sequence of phases that when they occur parties that are in different timelines can converge on the same single phase and make a unanimous decision to update their state.

\smallskip\noindent\emph{A new PoW-based permissionless fast SMR protocol.}
%
Given that Chain-King Consensus is a one-shot multi-valued Byzantine agreement protocol terminating in expected constant rounds, next we show how to build a state machine replication protocol on top of its sequential composition.
%
The resulting protocol achieves consistency and expected-constant-time liveness for \emph{all types} of transactions (including the conflicting ones).
%
This answers a question left open in previous work on PoW-based fast ledgers \cite{CCS:BKTFV19,TCC:FGKR20}, where fast settlement of transactions was offered only for non-conflicting transactions, thus making our ledger construction the first expected-constant processing time ledger in the PoW setting.
%
We note that fast processing of conflicting transactions can be crucial for many applications such as sequencing smart contract operations.
%
We also describe how it is possible to ``bootstrap from genesis'': this essential operation permits new parties to join the protocol execution as well as facilitate third party observers who wish to connect and parse the distributed ledger in order to issue transactions or read transaction outputs.

\subsection{Related Work}
\label{subsec:related-work}

\paragraph{Round complexity of synchronous BA protocols.}
%
For ``classical'' BA protocols with deterministic termination, it is known that $t + 1$ rounds \cite{InfProcessLett:FisLyn82} are necessary, where $t$ denotes the upper bound on the number of corrupted parties, and matching upper bounds exist, both in the information-theoretic and cryptographic settings~\cite{TOPLAS:LamShoPea82,SICOMP:DolStr83,STOC:GarMos93}.

The linear dependency of the number of rounds on the number of corrupted parties can be circumvented by introducing randomization.
%
Rabin \cite{FOCS:Rabin83} showed that consensus reduces to an ``oblivious common coin'' (OCC)---i.e., a common view of the honest parties of some public randomness.
%
As a result, randomized protocols with linear corruption resiliency and probabilistic termination in expected-constant rounds is possible.
%
Later on, Feldman and Micali \cite{STOC:FelMic88} showed how to construct an OCC ``from scratch'' and gave the first expected-constant-time Byzantine agreement protocol, tolerating the optimal number of corrupted parties (less than $1/3$ of the total number of parties), in the information-theoretic setting.
%
In the setting where trusted private setup (i.e., a PKI) is provided, Katz and Koo \cite{C:KatKoo06}  presented an expected-constant-round BA protocol with optimal resiliency (less than $1/2$ in the cryptographic setting).

We already mentioned that with the advent of blockchains, BA protocols that do not rely on a fixed set of participants became possible.
%
For PoW-based BA protocols, please refer to the beginning of this section.
%
Regarding Proof-of-Stake protocols, Algorand \cite{TCS:CheMic19} uses \emph{verifiable random functions} (VRFs) to self-elect parties, and agreement and validity are achieved in expected-constant time.

Regarding BA protocols based on some other assumptions, we note that in an unpublished manuscript (also mentioned in the introduction)~\cite{EPRINT:EckFauLos17}, Eckey, Faust and Loss design an expected-constant-round BA protocol based on PoWs and time-lock puzzles (TLPs).
%
Further, Das \textit{et al.} \cite{EPRINT:DEFLM22} propose a BA protocol based on the much stronger primitive of \emph{verifiable delay functions} (VDFs) that also terminate in expected-constant time.

\paragraph{Many PoWs from one PoW.}
%
As mentioned in the introduction, Garay, Kiayias and Leonardos \cite{EC:GarKiaLeo15} showed how to use a Nakamoto-style blockchain to solve BA.
%
Achieving the optimal corruption threshold of less than $1/2$ of the participants, however, presented some challenges, which were resolved by the introduction of a technique called ``\twoforone PoW,'' which is used to compose two modes of mining, one for blocks and one for inputs.
%
In a nutshell, in \twoforone PoW, a random oracle output is checked twice with respect to \emph{both} its leading zeros and tailing zeros.
%
Sufficient leading zeros implies the success of mining a block, and that's the original---i.e., Bitcoin's---approach to assess and verify whether a PoW has been produced, while sufficient \emph{trailing} zeros imply the success of mining an input.
%
This scheme guarantees that both mining procedures can be safely composed and the adversary is bound to its original computational power and is not able to favor one PoW operation over the other.

The \twoforone PoW primitive has found applications in many other scenarios (e.g.,~\cite{PODC:PasShi17}) and its generalization---\mforone PoW---makes parallel chains possible and has been used to improve transaction throughput \cite{CCS:BKTFV19} and for accelerating transaction confirmation \cite{TCC:FGKR20}.
%
We note that, in the case of parallel chains existing \mforone PoW constructions cannot achieve full independence on all parallel chains.
%
We elaborate on this in~\cref{sec:related-work-contd}.

\paragraph{Non-simultaneous termination and sequential composition.}
%
A consequence of the round complexity ``acceleration'' provided by randomized BA protocols is that their termination is probabilistic and not necessarily simultaneous \cite{JACM:DolReiStr90}.
%
This is problematic when this type of BA protocol is invoked by a higher-level protocol.
%
More specifically, parties would not be able to figure out when to safely return to the higher-level protocol and start the next execution.
%
One solution is to run randomized BA protocols for $\bigO(\polylog \kappa)$ rounds where $\kappa$ is the security parameter.
%
The running time is still independent of the number of parties, and, with overwhelming probability, parties would terminate and be able to start the next execution when $\bigO(\polylog \kappa)$ rounds have elapsed.
%
A more sophisticated sequential composition approach is to employ so-called ``Bracha termination'' and ``super-round'' expansion in order to preserve an expected-constant round complexity (cf.~\cite{C:CCGZ16}).
%
We adapt these techniques to the permissionless setting.

\paragraph{Settlement latency in state machine replication.}
%
Most PoW-based SMR protocols achieve liveness in a time which is a function of the security parameter, hence suffering from long transaction settlement latency.
%
The ``Ledger Combiner'' approach \cite{TCC:FGKR20} proposes a novel grade assignment function to build a virtual ledger on top of different parallel ledgers, achieving constant settlement time but only for \emph{non-conflicting} transactions.
%
Prism \cite{CCS:BKTFV19} also gives a PoW-based parallel chain protocol with expected-constant settlement time, but only for non-conflicting transactions.
%
Other approaches to fast transaction settlement include Algorand's~\cite{TCS:CheMic19}, which being Proof-of-Stake-based, achieves expected-constant settlement delay for all types of transactions.
%
Finally, Momose and Ren \cite{CCS:MomRen22} achieve expected-constant confirmation delay, assuming a PKI and VRFs.

