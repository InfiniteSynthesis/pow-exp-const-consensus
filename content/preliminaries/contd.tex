\section{Models and Preliminaries (Cont'd)}
\label{sec:model-preliminaries-contd}

\paragraph{Clock functionality.}
%
We adopt \funcClock (cf.~\cite{TCC:KMTZ13}) to model synchronous processors.
%
In a nutshell, \funcClock internally maintains a round variable $\tau$ which is only updatable when all parties send it the \textsc{clock-update} command.
%
Whenever a party \party is activated, \party sends a \textsc{clock-read} message to check the current round.
%
When round proceeds, \party executes the protocol and send \textsc{clock-update} after it completes all computations; if not, \party does nothing and wait for the next activation.
%
By interacting with \funcClock, parties are aware that they proceed in synchronized rounds.

\begin{cccFunctionality}
    {\funcClock}
    {clock}
    {The global clock.}

    This functionality maintains state variables as follows.

    \addtocounter{table}{-1}
    \begin{tabularx}{.9\textwidth}{c X}
        \toprule[.3mm]
        \textbf{State Variable}
         & \textbf{Description}
        \\ \midrule[.3mm]
        $\partyset \gets \emptyset$
         & The set of registered parties $\party = (\pid, \sid)$.
        \\ \midrule
        $F \gets \emptyset$
         & The set of registered functionalities (together with their session identifier).
        \\ \midrule
        $\tau_\sid \gets 0$
         & The clock variable for session \sid.
        \\ \midrule
        $d_\party \gets 0$
         & The clock-update variable for $\party = (\pid, \sid) \in \partyset$.
        \\ \midrule
        $d(\F, \sid) \gets 0$
         & The clock-update variable for $(\F, \sid) \in F$.
        \\ \bottomrule[.3mm]
    \end{tabularx}

    \begin{cccItemize}[noitemsep]
        \item Upon receiving $(\textsc{clock-update}, \sid_C)$ from some party $\party \in \partyset$ set $d_\party \gets 1$; execute \emph{Round-Update} and forward $(\textsc{clock-update}, \sid_C, \party)$ to \adv.

        \item Upon receiving $(\textsc{clock-update}, \sid_C)$ from some functionality \F in a session \sid such that $(\F, \sid) \in F$ set $d(\F, \sid) \gets 1$, execute \emph{Round-Update} and return $(\textsc{clock-update},\allowbreak \sid_C, \F)$ to this instance of \F.

        \item Upon receiving $(\textsc{clock-read}, \sid_C)$ from any participant (including the environment on behalf of a party, the adversary, or any ideal—shared or local—functionality) return $(\textsc{clock-read}, sid_C , \tau_{\sid})$ to the requestor (where \sid is the \sid of the calling instance).
    \end{cccItemize}

    \emph{Procedure Round-Update:} For each session \sid do: If $d(\F, \sid) = 1$ for all $\F \in F$ and $d_\party = 1$ for all honest partyset $P = (\cdot, \sid) \in \partyset$, then set $\tau_\sid \gets \tau_\sid + 1$ and reset $d(\F, \sid) \gets 0$ and $d_\party \gets 0$ for all partyset $\party = (\cdot, \sid) \in \partyset$.
\end{cccFunctionality}


\paragraph{Random oracle functionality.}
%
By convention, we model the hash function used to generate PoW as a random oracle; this is captured by the functionality \funcRO.

\begin{cccFunctionality}{\funcRO}
    {random-oracle}
    {The random oracle.}

    The functionality is parameterized by a security parameter $\kappa$.

    \begin{tabularx}{.9\textwidth}{c  X}
        \toprule[.3mm]
        \textbf{State Variable}
         & \textbf{Description}
        \\ \midrule[.3mm]
        $\partyset \gets \emptyset$
         & The set of registered parties.
        \\ \midrule
        $H \gets \emptyset$
         & A dynamically updatable function table where $H[x] = \bot$ denotes the fact that no pair of the form $(x, \cdot)$ is in $H$.
        \\ \bottomrule[.3mm]
    \end{tabularx}
    \addtocounter{table}{-1}

    \begin{cccItemize}[noitemsep]
        \item \textbf{Eval.} Upon receiving $(\textsc{eval}, \sid, x)$ from some party $\party \in \partyset$ (or from \adv on behalf of a corrupted \party), do the following:
        %
        \begin{cccEnum}[nosep]
            \item If $H[x] = \bot$ sample a value $y$ uniformly at random from $\{0, 1\}^\kappa$ and set $H[x] \gets y$.
            \item Return $(\textsc{eval}, \sid, x, H[x])$ to the requestor.
        \end{cccEnum}
    \end{cccItemize}
\end{cccFunctionality}


Note that with regards to bounding access to real-world resources, functionality \funcRO as defined fails to limit the adversary on making a certain number of queries per round.
%
Hence, we adopt a functionality wrapper \cite{C:BMTZ17,EC:GKOPZ20} $\mathcal{W}(\funcRO)$ that wraps the corresponding resource to capture such restrictions.

\begin{cccFunctionality}
    {$\mathcal{W}(\funcRO)$}
    {random-oracle-wrapper}
    {The random oracle wrapper.}

    This functionality maintains state variables as follows.

    \begin{minipage}{\linewidth}
        \begin{tabularx}{.9\textwidth}{c  X}
            \toprule[.3mm]
            \textbf{State Variable}
             & \textbf{Description}
            \\ \midrule[.3mm]
            $\partyset \gets \emptyset$
             & The set of registered parties; the current set of corrupted parties is denoted by $\partyset'$.
            \\ \midrule
            $\tau \gets 0$
             & The clock variable.
            \\ \midrule
            $h_\tau$
             & An upper bound which restricts the \func-evaluations of all honest parties at time $\tau$.
            \\ \midrule
            $q_\honestPartySet, q_\adv \gets 0$
             & The honest/adversary evaluation counter.
            \\ \bottomrule[.3mm]
        \end{tabularx}
        \addtocounter{table}{-1}
    \end{minipage}

    \medskip
    \paragraph{Pre-mining attack handling (executed only if $\tau = 0$):}
    %
    \begin{cccItemize}[nosep]
        \item Upon receiving $(\textsc{eval}, \sid, x)$ from \adv on behalf of a corrupted party $P \in \partyset'$, forward the request to \funcRO and return to \adv whatever \funcRO returns.

        \item Upon receiving $(\textsc{Retrieved}, \sid)$ from \funcCRS, set $\tau = 1$.
    \end{cccItemize}

    \paragraph{Relaying inputs to the random oracle:}
    %
    \begin{cccItemize}[nosep]
        \item Upon receiving $(\textsc{eval}, \sid, x)$ from \adv on behalf of a corrupted party $P \in \partyset'$, first execute \textit{Round Reset}, then do the following.
        %
        \begin{cccEnum}[nosep]
            \item Set $q_\adv \gets q_\adv + 1$.
            \item \textbf{If} $q_\adv < h_\tau$ \textbf{then} forward the request to \funcRO and return to \adv whatever \funcRO returns.
        \end{cccEnum}

        \item Upon receiving $(\textsc{eval}, \sid, x)$ from an honest party \party, first execute \textit{Round Reset}, then do the following.
        %
        \begin{cccEnum}[nosep]
            \item Set $q_\honestPartySet \gets q_\honestPartySet + 1$.
            \item \textbf{If} $q_\honestPartySet \le h_\tau$ \textbf{then} forward the request to \funcRO and return to \party whatever \funcRO returns.
            \item \textbf{If} $q_\honestPartySet \ge h_\tau$ \textbf{then} send $(\textsc{clock-update}, \sid_C)$ to \funcClock.
        \end{cccEnum}
    \end{cccItemize}

    \paragraph{Corruption handling:}
    %
    \begin{cccItemize}[nosep]
        \item Upon receiving $(\textsc{corrupt}, \sid, \party)$ from the adversary, set $\partyset' \gets \partyset' \cup \party$.
    \end{cccItemize}

    \medskip\emph{Procedure Round-Reset:}
    %
    Send $(\textsc{clock-read}, \sid_C)$ to \funcClock and receive $(\textsc{clock-read}, \allowbreak \sid_C, \tau')$ from \funcClock. If $|\tau - \tau' | > 0$, then set $q_\honestPartySet, q_\adv \gets 0$ and $\tau \gets \tau'$.
\end{cccFunctionality}


\paragraph{Diffusion functionality.}
%
We model the synchronous communication by \funcDiffuse \cite{C:BMTZ17}.
%
Note that we present $\funcDiffuse^\delay$ which is parameterized by the network delay \delay, and the synchronous variant can be easily derived by setting $\delay = 1$.

\begin{cccFunctionality}
      {\funcDiffuse}
      {diffuse}
      {The diffusion network.}

      \newcommand*{\msgid}{\ensuremath{\mathsf{mid}}\xspace}
      \newcommand*{\vecM}{\ensuremath{\vec{M}}\xspace}

      The functionality is parameterized by the network delay \delay.

      \addtocounter{table}{-1}
      \begin{tabularx}{.9\textwidth}{c  X}
            \toprule[.3mm]
            \textbf{State Variable}
             & \textbf{Description}
            \\ \midrule[.3mm]
            $\partyset \gets \emptyset$
             & The set of registered parties.
            \\ \midrule
            $\vecM \gets [] $
             & A dynamically updatable list of quadruples $(m, \msgid, D_{\msgid}, \party)$ where $D_{\msgid}$ denotes the fetch counter.
            \\ \bottomrule[.3mm]
      \end{tabularx}

      \paragraph{Network capabilities:}
      %
      \begin{cccItemize}[nosep]
            \item Upon receiving $(\textsc{diffuse}, \sid, m)$ from some
            $\party_s \in \partyset$, where $\partyset = \{ \party_1, \ldots, \party_n \}$ denotes the current party set, do:
            %
            \begin{cccEnum}[nosep]
                  \item Choose $n$ new unique message-IDs $\msgid_1, \ldots, \msgid_n$.
                  \item Initialize $2n$ new variables $D_{\msgid_1} := D^{MAX}_{\msgid_1} \ldots := D_{\msgid_n} := D^{MAX}_{\msgid_n} := 1$ and a per message-delay $\delay_{\msgid_i} = \delay$ for $i \in [n]$.
                  \item Set  $\vecM := \vecM \concat (m, \msgid_1, D_{\msgid_1}, \party_1) \concat \ldots \concat (m, \msgid_n, D_{\msgid_n}, \party_n)$.
                  \item Send $(\textsc{diffuse}, \sid, m, \party_s, (\party_1, \msgid_1), \ldots ,(\party_n, \msgid_n))$ to the adversary.
            \end{cccEnum}

            \item Upon receiving $(\textsc{fetch}, \sid)$ from $\party \in \partyset$, or from \adv on behalf of a corrupted party \party:
            %
            \begin{cccEnum}[nosep]
                  \item For all tuples $(m, \msgid, D_\msgid, \party) \in \vecM$, set $D_\msgid := D_\msgid - 1$.
                  \item Let $\vecM^\party_0$ denote the subvector \vecM including all tuples of the form $(m, \msgid, D_\msgid, \party)$ with $D_\msgid \le 0$ (in the same order as they appear in \vecM).
                  %
                  Delete all entries in $\vecM^\party_0$ from \vecM and in case some $(m, \msgid, D_\msgid, \party)$ is in
                  $\vecM^\party_0$, where \party is honest, set $\delay_{\msgid'} = \delay$ for any $(m, \msgid', D_{\msgid'}, (\cdot, \sid))$ in \vecM and replace this record by $(m, \msgid', \min \{ D_{\msgid'}, \delay \}, \party')$.
                  \item Output $\vecM^\party_0$ to \party (if \party is corrupted, send $\vecM^\party_0$ to \adv).
            \end{cccEnum}
      \end{cccItemize}

      \paragraph{Additional adversarial capabilities:}
      %
      \begin{cccItemize}[nosep]
            \item Upon receiving $(\textsc{diffuse}, \sid, m)$ from some corrupted $\party_s \in \partyset$ (or from \adv on behalf of $\party_s$ if corrupted),
            % do 
            execute it the same way as an honest-sender diffuse, with the only difference that $\delay_{\msgid_i} = \infty$.

            \item Upon receiving $(\textsc{delays}, \sid, (T_{\msgid_{i_1}}, \msgid_{i_1}), \ldots, (T_{\msgid_{i_\ell}}, \msgid_{i_\ell}))$ from the adversary do the following for each pair $(T_{\msgid_{i_j}}, \msgid_{i_j})$:
            %
            if $D^{MAX}_{\msgid_{i_j}} + T_{\msgid_{i_j}} \le \delay_{\msgid_{i_j}}$ and $\msgid_{i_j}$ is a message-ID of receiver $\party = (\cdot, \sid)$ registered in the current \vecM, set $D_{\msgid_{i_j}} := D_{\msgid_{i_j}} + T_{\msgid_{i_j}}$ and set $D^{MAX}_{\msgid_{i_j}} := D^{MAX}_{\msgid_{i_j}} + T_{\msgid_{i_j}}$; otherwise, ignore this pair.

            \item Upon receiving $(\textsc{swap}, \sid, \msgid, \msgid')$ from the adversary, if \msgid and $\msgid'$ are message-IDs registered in the current \vecM, then swap the triples $(m, \msgid, D_\msgid, (\cdot, \sid))$ and $(m, \msgid', D_{\msgid'}, (\cdot, \sid))$ in \vecM.
            %
            Return $(\textsc{swap}, \sid)$ to the adversary.

            \item Upon receiving $(\textsc{get-reg}, \sid)$ from \adv, return the response $(\textsc{get-reg}, \sid, \partyset)$ to \adv.
      \end{cccItemize}
\end{cccFunctionality}

\paragraph{Common reference string functionality.}
%
We model a public setup by the CRS functionality \funcCRS.

\begin{cccFunctionality}
    {\funcCRS}
    {CRS}
    {The common reference string.}

    The functionality is parameterized by a distribution $\mathcal{D}$.

    \begin{cccItemize}[noitemsep]
        \item \textbf{Retrieve.} Upon receiving $(\textsc{Retrieve}, \sid)$ from some party \party (or from \adv on behalf of a corrupted \party), do the following:
        %
        \begin{cccEnum}[nosep]
            \item If activated for the first time, choose a value $d \gets \mathcal{D}$, and send $(\textsc{Retrieved}, \sid)$ to $\mathcal{W}(\funcRO)$.

            \item Return $(\textsc{Retrieve}, d)$ to \party.
        \end{cccEnum}
    \end{cccItemize}
\end{cccFunctionality}
