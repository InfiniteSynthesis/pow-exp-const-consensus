\subsection{Proof of~\cref*{thm:typical-execution-parallel-chain}}
\label{subsec:typical-exeuction-parallel-chains}

In this section we present a formal analysis of the security of parallel chains by using the typical execution analytical framework from \cite{EPRINT:GarKiaLeo14,C:GarKiaLeo17}.

\paragraph{Notations and preliminaries.}
%
We write $h$ as the number of honest RO queries and $t$ as the number of corrupted RO queries.
%
To illustrate the honest advantage against the adversary, we adopt an additional parameter $\delta \in (0, 1]$ such that $t \le (1 - \delta) h$.
%
In order for our protocol to work, the following conditions should be satisfied.
%
\begin{equation} \tag{C1} \label{condition:delta-advantage}
    3f + 3\epsilon < \delta \le 1.
\end{equation}

For the purpose of estimating the number of blocks acquired by honest parties during a sequence of rounds, we define the following random variables $X, X', Y, Z, Z'$ with respect to round $r$.
%
First, if in round $r$ \emph{at least} one honest party successfully solves a PoW with respect to chain-block, then $X_r = 1$ (we call $r$ a successful round); otherwise $X_r = 0$.
%
If in round $r$ \emph{at least} one honest party successfully solves a PoW with respect to input-block, then $X'_r = 1$; otherwise $X'_r = 0$.
%
If in round $r$ \emph{exactly} one honest party solves a PoW w.r.t. chain-block, then $Y_r = 1$ (we call $r$ a unique successful round); otherwise $Y_r = 0$.
%
Regarding the adversary, let $Z_r$ ($Z'_r$ resp.) denote the total number of successful PoWs with respect o chain-blocks (input-blocks resp.) that the adversary gets at round $r$.
%
Note that $Z_r$ ($Z'_r$ resp.) can be viewed as the sum of Bernoulli random variable $Z_{rij}$ ($Z'_{rij}$ reps.) denote the $i$-th query of the $j$-th corrupted party in round $r$.
%
For a set of rounds $S$, let $X(S) = \sum_{r \in S} X_r$ and similarly define $X'(S), Y(S), Z(S)$ and $Z'(S)$.

The following mathematical facts with respect to $X_r, X'_r, Y_r, Z_r, Z'_r$, derived from the $q$-bounded model \cite{EPRINT:GarKiaLeo14} to one query per party per round, can be useful in later proofs.
%
\begin{align*}
    (1 - f) p h < f = & \EX[X_r] = \EX[X'_r] = 1 - (1 - p)^{h} < p h,
    \\
                      & \EX[Y_r] \ge p h (1 - p)^{h - 1} > p h [1 - p h] \ge f (1 - f),
    \\
                      & \EX[Z_r] = \EX[Z'_r] = p t = \frac{t}{h} p h < \frac{t}{h} \frac{f}{1 - f} < (1 + \frac{\delta}{2}) \cdot f \cdot \frac{t}{h}.
\end{align*}

\paragraph{Typical executions.}
%
We formally define typical executions.
%
An execution is typical if for any set $S$ of at least $k$ consecutive rounds, it holds that (i) for all random variables $X(S), X'(S), Y(S), Z(S)$ and $Z'(S)$, the deviation from their expected value is bounded by a concentration quality parameter $\epsilon$; and (ii) no bad events with respect to random oracle (collisions) happen during these rounds.

Regarding bad events with respect to random oracle (cf.~\cite{EPRINT:GarKiaLeo14}), an \emph{insertion} occurs when, given a chain \chain with two consecutive blocks \block and $\block'$, a block $\block^*$ created after $\block'$ is such that $\block, \block^*, \block'$ form three consecutive blocks of a valid chain; a \emph{copy} occurs if the same block exists in two different positions; a \emph{prediction} occurs when a block extends one which was computed at a later round.

\begin{definition}
    [Typical execution]
    \label{def:typical-execution}

    An execution is $(\epsilon, k)$-typical, for $\epsilon \in (0, 1)$ and constant integer $k$ if, for any set $S$ of at least $k$ consecutive rounds, the following hold.
    %
    \begin{enumerate}[label=(\alph*), leftmargin=*, nosep]
        \item $(1 - \epsilon) \EX[X(S)] < X(S) < (1 + \epsilon) \EX[X(S)]$ and $(1 - \epsilon) \EX[X'(S)] < X'(S) < (1 + \epsilon) \EX[X'(S)]$

        \item $(1 - \epsilon) \EX[Y(S)] < Y(S)$.

        \item $Z(S) < \EX[Z(S)] + \epsilon \EX[X(S)]$ and $Z'(S) < \EX[Z'(S)] + \epsilon \EX[X'(S)]$.

        \item No insertions, no copies, and no predictions occurred.
    \end{enumerate}
\end{definition}

With typical executions and random variables defined above, we prove the following lemma.

\begin{lemma} \label{lemma:typical-execution-single-chain}
    For any $\beta < 1$, there exist protocol parameterizations such that an execution running for a constant number \phaseLength of rounds is typical with probability at least $\beta$.
\end{lemma}

\begin{proof}
    Regarding Property (a), (b) and (c), consider any $k$ consecutive rounds.
    %
    Let $\mathsf{badX}$ denote the event that either $X(S) \ge (1 + \epsilon) \EX[X(S)]$ or $X(S) \le (1 - \epsilon) \EX[X(S)]$; let $\mathsf{badY}$ denote the event that $Y(S) \le (1 - \epsilon) \EX[Y(S)]$; and let $\mathsf{badZ}$ denote the event that $Z(S) \ge \EX[Z(S)] + \epsilon \EX[X(S)]$.
    %
    Following the Chernoff Bounds (\cref{thm:chernoff-bounds}), we have
    %
    \begin{align*}
        \Pr[\mathsf{badX}] = \Pr[\mathsf{badX}'] & \le \exp(-\epsilon f k / 3) + \exp(-\epsilon f k / 2),
        \\
        \Pr[\mathsf{badY}]                       & \le \exp(-\epsilon f k / 2),
        \\
        \Pr[\mathsf{badZ}] = \Pr[\mathsf{badZ}'] & \le \exp(-2\epsilon^2 f^2 k^3).
    \end{align*}
    %
    The probability that neither one of them happens is lower-bounded by
    %
    \[ \Pr \big[\neg \mathsf{badX} \wedge \neg \mathsf{badX}' \wedge \neg \mathsf{badY} \wedge \neg \mathsf{badZ} \wedge \neg \mathsf{badZ}' \big] \ge 1 - 7\exp \big( -\epsilon f k / 3 \big). \]
    %
    Notice that $k \mapsto 1 - 7 \exp(-\epsilon f k / 3)$ is increasing in $(0, +\infty)$, and $\lim_{k \rightarrow \infty} 1 - 7 \exp(-\epsilon f k / 3) = 1$, for any $\beta' < 1$ there exists $k$ such that $\Pr[\neg \mathsf{badX} \wedge \neg \mathsf{badX}' \wedge \neg \mathsf{badY} \wedge \neg \mathsf{badZ} \wedge \neg \mathsf{badZ}'] \ge \beta'$.

    We then consider the execution running for \phaseLength rounds.
    %
    Without loss of generality, assume $\phaseLength = c \cdot k$ for a constant $c \in \mathbb{N}^+$.
    %
    The number of set of at least $k$ consecutive rounds in \phaseLength is $(1/2)(ck - k +2)(ck - k +1) < c^2k^2$.
    %
    Hence, the probability that an execution running for $\phaseLength = c \cdot k$ steps and Property (a)(b)(c) holds on any set of $k$ consecutive rounds is lower-bounded by
    %
    \[ [1 - 7\exp(-\epsilon f k / 3)]^{c^2 k^2} \ge 1 - 7c^2 k^2 \exp(-\epsilon f k / 3) = 1 - 7\exp(-\epsilon f k / 3 + 2\ln c + 2\ln k). \]
    %
    Notice that there exists constant $c$ such that $k \mapsto 1 - 7\exp(-\epsilon f k / 3 + 2\ln c + 2\ln k)$ is increasing in $(c, +\infty)$ and $\lim_{k \rightarrow \infty} 1 - 7\exp(-\epsilon f k / 3 + 2\ln c + 2\ln k) = 1$, for any $\beta < 1$ there exists $k$ such that the probability that an execution running for $\phaseLength = c \cdot k$ steps and Property (a)(b)(c) holds on any set of $k$ consecutive rounds is at least $\beta$.

    Regarding Property (d), the RO queries made by all parties in \phaseLength rounds is $Q = (h + t) \cdot \phaseLength$.
    %
    Recall our \mforone PoW construction in~\cref{subsec:parallel-chains}, we consider RO output of length $\polylog \kappa$.
    %
    A collision happens with probability $Q^2 / 2^{\polylog \kappa} = \exp(-\Omega(\polylog \kappa + \log \phaseLength))$ which is negligible.
    %
    (Note that Property (d) actually holds on an execution of $L = \poly (\kappa)$ steps.)
\end{proof}

Recall that we run $m = \Theta(\log^2 \kappa)$ chains in parallel using \mforone PoW, and all parallel executions are mutually independent of each other, we show that the above probability on a single execution translates to the fraction of typical executions among $m$ parallel ones.

\theoremtypicalexeuctionparallelchain*

\begin{proof}
    Fix $\beta < 1$.
    %
    We define random variable $Q_i (i \in [m])$ as follows.
    %
    If the $i$-th execution among $m$ parallel repetitions, running for a constant number of \phaseLength steps is typical, $Q_i = 1$; otherwise $Q_i = 0$.
    %
    Also let $Q = \sum Q_i$.
    %
    From~\cref{lemma:typical-execution-single-chain} we learn that there exists protocol parameterizations such that $\forall i \in [m], \Pr[Q_i = 1] = (1 + 2\epsilon)\beta$.
    %
    Hence $\EX[Q] = (1 + 2\epsilon)\beta m$.
    %
    Since our \mforone PoW construction achieves full independence over all parallel chains, we have
    %
    \[ \Pr[Q \le \beta m] < \Pr[Q \le (1 - \epsilon) (1 + 2\epsilon) \beta m] = \Pr[Q \le (1 - \epsilon) \EX[Q]] \le \exp (- \epsilon^2 (1 + 2\epsilon) \beta m). \]
    %
    The first inequality holds because $\epsilon < 1/2$, and the last one is due to the Chernoff bounds.
    %
    Since $m = \Theta(\log^2 \kappa)$, this probability is negligible in terms of $\kappa$.
    %
    I.e., there exist protocol parameterizations such that, with overwhelming probability, at least $\beta$ fraction of the executions are typical.
\end{proof}
