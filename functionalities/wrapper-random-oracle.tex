\begin{cccFunctionality}
    {$\mathcal{W}(\funcRO)$}
    {random-oracle-wrapper}
    {The random oracle wrapper.}

    This functionality maintains state variables as follows.

    \begin{minipage}{\linewidth}
        \begin{tabularx}{.9\textwidth}{c  X}
            \toprule[.3mm]
            \textbf{State Variable}
             & \textbf{Description}
            \\ \midrule[.3mm]
            $\partyset \gets \emptyset$
             & The set of registered parties; the current set of corrupted parties is denoted by $\partyset'$.
            \\ \midrule
            $\tau \gets 0$
             & The clock variable.
            \\ \midrule
            $h_\tau$
             & An upper bound which restricts the \func-evaluations of all honest parties at time $\tau$.
            \\ \midrule
            $q_\honestPartySet, q_\adv \gets 0$
             & The honest/adversary evaluation counter.
            \\ \bottomrule[.3mm]
        \end{tabularx}
        \addtocounter{table}{-1}
    \end{minipage}

    \medskip
    \paragraph{Pre-mining attack handling (executed only if $\tau = 0$):}
    %
    \begin{cccItemize}[nosep]
        \item Upon receiving $(\textsc{eval}, \sid, x)$ from \adv on behalf of a corrupted party $P \in \partyset'$, forward the request to \funcRO and return to \adv whatever \funcRO returns.

        \item Upon receiving $(\textsc{Retrieved}, \sid)$ from \funcCRS, set $\tau = 1$.
    \end{cccItemize}

    \paragraph{Relaying inputs to the random oracle:}
    %
    \begin{cccItemize}[nosep]
        \item Upon receiving $(\textsc{eval}, \sid, x)$ from \adv on behalf of a corrupted party $P \in \partyset'$, first execute \textit{Round Reset}, then do the following.
        %
        \begin{cccEnum}[nosep]
            \item Set $q_\adv \gets q_\adv + 1$.
            \item \textbf{If} $q_\adv < h_\tau$ \textbf{then} forward the request to \funcRO and return to \adv whatever \funcRO returns.
        \end{cccEnum}

        \item Upon receiving $(\textsc{eval}, \sid, x)$ from an honest party \party, first execute \textit{Round Reset}, then do the following.
        %
        \begin{cccEnum}[nosep]
            \item Set $q_\honestPartySet \gets q_\honestPartySet + 1$.
            \item \textbf{If} $q_\honestPartySet \le h_\tau$ \textbf{then} forward the request to \funcRO and return to \party whatever \funcRO returns.
            \item \textbf{If} $q_\honestPartySet \ge h_\tau$ \textbf{then} send $(\textsc{clock-update}, \sid_C)$ to \funcClock.
        \end{cccEnum}
    \end{cccItemize}

    \paragraph{Corruption handling:}
    %
    \begin{cccItemize}[nosep]
        \item Upon receiving $(\textsc{corrupt}, \sid, \party)$ from the adversary, set $\partyset' \gets \partyset' \cup \party$.
    \end{cccItemize}

    \medskip\emph{Procedure Round-Reset:}
    %
    Send $(\textsc{clock-read}, \sid_C)$ to \funcClock and receive $(\textsc{clock-read}, \allowbreak \sid_C, \tau')$ from \funcClock. If $|\tau - \tau' | > 0$, then set $q_\honestPartySet, q_\adv \gets 0$ and $\tau \gets \tau'$.
\end{cccFunctionality}
